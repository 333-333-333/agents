\documentclass[11pt, a4paper]{article}

% ============================================
% Packages
% ============================================
\usepackage[utf8]{inputenc}
\usepackage[spanish]{babel}
\usepackage{geometry}
\usepackage{xcolor}
\usepackage{hyperref}
\usepackage{fancyhdr}
\usepackage{graphicx}
\usepackage{booktabs}
\usepackage{caption}
\usepackage{parskip}
\usepackage{tabularx}
\usepackage{enumitem}
\usepackage{multirow}
\usepackage{colortbl}
\usepackage{amsmath}
\usepackage{tikz}
\usepackage{tcolorbox}

% ============================================
% Color definitions — Catppuccin Latte (print)
% Ref: skills/corporate-colors/SKILL.md
% ============================================

% -- Primary --
\definecolor{primary}{HTML}{1e66f5}
\definecolor{primaryHover}{HTML}{04a5e5}
\definecolor{primaryActive}{HTML}{7287fd}

% -- Secondary --
\definecolor{secondary}{HTML}{8839ef}
\definecolor{secondaryHover}{HTML}{ea76cb}

% -- Accent --
\definecolor{success}{HTML}{40a02b}
\definecolor{warning}{HTML}{df8e1d}
\definecolor{error}{HTML}{d20f39}
\definecolor{info}{HTML}{209fb5}

% -- Surface --
\definecolor{bgBase}{HTML}{eff1f5}
\definecolor{bgMantle}{HTML}{e6e9ef}
\definecolor{bgCrust}{HTML}{dce0e8}

% -- Text --
\definecolor{textPrimary}{HTML}{4c4f69}
\definecolor{textSecondary}{HTML}{5c5f77}
\definecolor{textTertiary}{HTML}{6c6f85}

% -- Border --
\definecolor{border}{HTML}{acb0be}
\definecolor{borderSubtle}{HTML}{bcc0cc}
\definecolor{borderEmphasis}{HTML}{8839ef}

% ============================================
% Page layout
% ============================================
\geometry{margin=2.5cm, headheight=22pt, headsep=12pt}
\setlength{\parindent}{0pt}
\setlength{\parskip}{0.8em}

% ============================================
% Global text color
% ============================================
\color{textPrimary}

% ============================================
% Header / Footer
% ============================================
\pagestyle{fancy}
\fancyhf{}
\fancyhead[L]{\small\textcolor{textSecondary}{TÍTULO DEL DOCUMENTO}}
\fancyhead[R]{\small\textcolor{textSecondary}{\today}}
\fancyfoot[C]{\small\textcolor{textTertiary}{\thepage}}
\renewcommand{\headrulewidth}{0.4pt}
\renewcommand{\headrule}{\hbox to\headwidth{%
  \color{border}\leaders\hrule height \headrulewidth\hfill}}
\renewcommand{\footrulewidth}{0pt}

% ============================================
% Section styling
% ============================================
\usepackage{titlesec}
\titleformat{\section}
  {\Large\bfseries\color{primary}}
  {\thesection.}{0.5em}{}
\titleformat{\subsection}
  {\large\bfseries\color{primary!80!textPrimary}}
  {\thesubsection.}{0.5em}{}
\titleformat{\subsubsection}
  {\normalsize\bfseries\color{primary!60!textPrimary}}
  {\thesubsubsection.}{0.5em}{}

% ============================================
% Hyperlinks
% ============================================
\hypersetup{
  colorlinks=true,
  linkcolor=primary,
  urlcolor=primary,
  citecolor=secondary,
  pdftitle={Título del Documento},
  pdfauthor={Autor},
}

% ============================================
% Table styling
% ============================================
\arrayrulecolor{border}
\captionsetup{
  font={small, color=textSecondary},
  labelfont={bf, color=primary},
}

% ============================================
% Custom macros
% ============================================

% Currency
\newcommand{\clp}[1]{\$#1~CLP}
\newcommand{\usdtoclp}[3]{USD~\$#1 (~\clp{#2} al tipo de cambio de \$#3~CLP/USD)}

% Callout boxes
\newtcolorbox{infobox}{
  colback=info!5,
  colframe=info,
  fonttitle=\bfseries\color{info},
  title=Información,
  boxrule=0.5pt,
  arc=4pt,
}

\newtcolorbox{warningbox}{
  colback=warning!5,
  colframe=warning,
  fonttitle=\bfseries\color{warning},
  title=Advertencia,
  boxrule=0.5pt,
  arc=4pt,
}

\newtcolorbox{errorbox}{
  colback=error!5,
  colframe=error,
  fonttitle=\bfseries\color{error},
  title=Crítico,
  boxrule=0.5pt,
  arc=4pt,
}

\newtcolorbox{successbox}{
  colback=success!5,
  colframe=success,
  fonttitle=\bfseries\color{success},
  title=Logrado,
  boxrule=0.5pt,
  arc=4pt,
}

% ============================================
% Version control table
% ============================================
\newcommand{\versiontable}[1]{%
  \begin{center}
  \small
  \begin{tabularx}{\textwidth}{|l|l|l|X|}
    \hline
    \rowcolor{bgMantle}
    \textbf{\textcolor{primary}{Versión}} &
    \textbf{\textcolor{primary}{Fecha}} &
    \textbf{\textcolor{primary}{Autor}} &
    \textbf{\textcolor{primary}{Cambios}} \\
    \hline
    #1
  \end{tabularx}
  \end{center}
  \vspace{1em}
}

\newcommand{\versionrow}[4]{%
  #1 & #2 & #3 & #4 \\ \hline
}

% ============================================
% Document metadata
% ============================================
\title{\textcolor{primary}{\textbf{Título del Documento}}}
\author{\textcolor{textSecondary}{Nombre del Autor}}
\date{\textcolor{textTertiary}{\today}}

% ============================================
% Begin document
% ============================================
\begin{document}

\maketitle
\thispagestyle{fancy}

% ============================================
% Control de versiones
% ============================================
\versiontable{
  \versionrow{1.0}{Enero 2026}{Nombre del Autor}{Versión inicial del documento}
}

% ============================================
% Resumen Ejecutivo
% ============================================
\section*{Resumen Ejecutivo}

% Propósito y alcance del documento
Describir brevemente el objetivo del documento, los temas cubiertos
y las conclusiones o recomendaciones principales.

\tableofcontents
\newpage

% ============================================
% 1. Contexto
% ============================================
\section{Contexto}

% Antecedentes y motivación del proyecto
Describir el contexto que origina este documento, incluyendo
antecedentes relevantes y la necesidad que se busca resolver.

% Alcance y limitaciones del análisis
Definir qué cubre y qué no cubre este documento.

% ============================================
% 2. Análisis
% ============================================
\section{Análisis}

% Descripción de la metodología utilizada
Explicar el enfoque metodológico...

\subsection{Hallazgos Principales}

% Resumen de los hallazgos más relevantes
Presentar los resultados clave del análisis.

% Evidencia cuantitativa que respalda los hallazgos
Incluir datos, tablas o gráficos.

\begin{table}[h]
\centering
\caption{Ejemplo de tabla con costos estimados}
\begin{tabularx}{\textwidth}{Xlrr}
\toprule
\rowcolor{bgMantle}
\textbf{Ítem} & \textbf{Unidad} & \textbf{Cantidad} & \textbf{Costo (CLP)} \\
\midrule
Desarrollo       & Horas & 160 & \$4.800.000 \\
Infraestructura  & Mes   & 12  & \$5.400.000 \\
Licencias        & Año   & 1   & \$250.000 \\
\midrule
\textbf{Total}   &       &     & \textbf{\$10.450.000} \\
\bottomrule
\end{tabularx}
\end{table}

% ============================================
% 3. Impacto Económico
% ============================================
\section{Impacto Económico}

% Inversión total requerida en CLP
Detallar la inversión total...

% Retorno esperado y plazo de recuperación
Describir el ROI esperado...

\begin{infobox}
% Nota sobre vigencia de datos económicos
Valores económicos calculados con datos de enero 2026.
Verificar vigencia antes de utilizar para decisiones financieras.
\end{infobox}

% ============================================
% 4. Conclusiones
% ============================================
\section{Conclusiones}

% Síntesis de hallazgos y viabilidad del proyecto
Resumir las conclusiones principales...

% Recomendaciones concretas y próximos pasos
Listar las acciones recomendadas.

\begin{enumerate}[leftmargin=2em]
  \item Primera recomendación.
  \item Segunda recomendación.
  \item Tercera recomendación.
\end{enumerate}

% ============================================
% Anexos (opcional)
% ============================================
% \newpage
% \appendix
% \section{Anexo A: Datos de Soporte}
% Incluir tablas extendidas, código fuente, o referencias adicionales.

\end{document}
